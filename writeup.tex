\documentclass[12pt,letterpaper]{article}
\usepackage{amsmath,amsthm,amsfonts,amssymb,amscd}
\usepackage{fullpage}
\usepackage{lastpage}
\usepackage{enumerate}
\usepackage{fancyhdr}
\usepackage{mathrsfs}
\usepackage{xcolor}
\usepackage[margin=3cm]{geometry}
\setlength{\parindent}{0.0in}
\setlength{\parskip}{0.05in}

% Edit these as appropriate
\newcommand\course{CS 476/676}
\newcommand\semester{Fall 2013}     % <-- current semester
\newcommand\hwnum{1}                  % <-- homework number
\newcommand\yourname{David Snyder} % <-- your name
\newcommand\login{Hopkins ID: 38C8F8}       
\newcommand\hwdate{Due: October 2nd 2013}           % <-- HW due date

\newenvironment{answer}[1]{
  \subsubsection*{Problem #1}
}


\pagestyle{fancyplain}
\headheight 35pt
\lhead{\yourname\ \login\\\course\ --- \semester}
\chead{\textbf{\Large Homework \hwnum}}
\rhead{\hwdate}
\headsep 10pt

\begin{document}

\noindent \emph{Homework Notes:} I collaborated with Nely Jimenez on this assignment. \\
\begin{answer}{1a}

At each millisecond we compute the current posterior given the likelihood of that data point and the prior. This posterior becomes the prior for the next timestep. We rely on equations (7.55-7.60) of Murphy.\\ \\
After observing $n$ data points, the probability of ${\mathbf{w}}$ at time $n+1$ is, 
\begin{eqnarray*}
  p(\mathbf{w}) = \mathcal{N}(\mathbf{w} \mid {\mathbf{w}}_{n}, {\mathbf{\Sigma}}_{n}).
\end{eqnarray*}
Where 
\begin{eqnarray*}
  {\mathbf{w}}_{n} = {\mathbf{\Sigma}}_{n}  {{\mathbf{\Sigma}}_{0}}^{-1}  {\mathbf{w}}_{0} + \frac{1}{{\sigma}^{2}}  {\mathbf{\Sigma}}_{n} {{\mathbf{X}}_{N}}^{T} \mathbf{y} \\
  {\mathbf{\Sigma}}_{n} = {\sigma}^{2} ({\sigma}^{2} {{\mathbf{\Sigma}}_{0}}^{-1} + {\mathbf{X}}^{T} \mathbf{X} )^{-1}
\end{eqnarray*}
You can write aligned equations as follows:
\begin{eqnarray*}
a &\sim& p(a)\\
b &\sim& p(b).
\end{eqnarray*}

You can write inline equations: $a \sim p(b)$, or one line equations:
\[
a \sim p(a).
\]

You can also add bullet points:

\begin{enumerate}[(a)]
\item
This is the first bullet point. It can be the solution to the first part of the question
\item
This is the solution to the second part of the question.
\end{enumerate}

You can also add figures (Figure~\ref{fig:example}).

\begin{figure}[htbp] %  figure placement: here, top, bottom, or page
   \centering
  % \includegraphics[width=2in]{name.pdf} % uncomment this line and put the figure in the same folder as this document.
   \caption{example caption}
   \label{fig:example}
\end{figure}

And tables (Table~\ref{tab:booktabs})

% Requires the booktabs if the memoir class is not being used
\begin{table}[htbp]
   \centering
   \caption{Table captions are better up top} 
   \begin{tabular}{|c|c|c|}\hline % Column formatting
         Animal    & Description & Price (\$)\\\hline
      Gnat      & per gram & 13.65 \\
                & each     &  0.01 \\
      Gnu       & stuffed  & 92.50 \\
      Emu       & stuffed  & 33.33 \\
      Armadillo & frozen   &  8.99 \\ \hline
   \end{tabular}
   \label{tab:booktabs}
\end{table}

\end{answer}

\begin{answer}{2} 

\end{answer}

\begin{answer}{3}
\end{answer}

\begin{answer}{4}

\end{answer}


\begin{answer}{5}
\end{answer}


\end{document}
