\documentclass{article}
\title{MLCD Homework 3}
\date{December 5th 2013}
\author{David Snyder, dsnyde29
        \and Adi, adi ID}

\usepackage{amsmath,amsthm,amssymb}

\begin{document}
\maketitle

\section*{Variation Inference on a Simple Network}
\subsection*{2.1.a}
TODO Derive meanfield
\subsection*{2.1.b}
See inference.R for an implementation of the meanfield update equations.

TODO: Does the approximation look reasonable? What does the KL divergence mean?

\subsection*{2.1.c}
TODO Derive struct meanfield
\subsection*{2.1.d}
See inference.R for an implementation of the structured meanfield
update equations. 

TODO:
Does this look like a better approx than mean-field?
What does the KL divergence mean in this case?
Give a brief explanation why structured mean field performs 
better for this bayesian network

\section*{Collapsed Gibbs Sampler}
The collapsed gibbs sampler was implemented and a shell script to run the sampler is provided.
./collapsed-sampler "input train file" "input test file" "output file preffix" "number of topics" "lambda" "alpha" "beta" "iterations" "burn-in"

Using this we generated the following output files, using different values for K (5 and 25), lambda (0.5, 0.8, 0.2)
and alpha(0.1 and 1.0). 1100 iterations were completed with 1000 burn in iterations.
collapsed-output-25-0.2-0.1.txt-phi
collapsed-output-25-0.2-0.1.txt-phi0
collapsed-output-25-0.2-0.1.txt-phi1
collapsed-output-25-0.2-0.1.txt-testll
collapsed-output-25-0.2-0.1.txt-theta
collapsed-output-25-0.2-0.1.txt-trainll
collapsed-output-25-0.5-0.1.txt-phi
collapsed-output-25-0.5-0.1.txt-phi0
collapsed-output-25-0.5-0.1.txt-phi1
collapsed-output-25-0.5-0.1.txt-testll
collapsed-output-25-0.5-0.1.txt-theta
collapsed-output-25-0.5-0.1.txt-trainll
collapsed-output-25-0.5-1.txt-phi
collapsed-output-25-0.5-1.txt-phi0
collapsed-output-25-0.5-1.txt-phi1
collapsed-output-25-0.5-1.txt-testll
collapsed-output-25-0.5-1.txt-theta
collapsed-output-25-0.5-1.txt-trainll
collapsed-output-5-0.5-0.1.txt-phi
collapsed-output-5-0.5-0.1.txt-phi0
collapsed-output-5-0.5-0.1.txt-phi1
collapsed-output-5-0.5-0.1.txt-testll
collapsed-output-5-0.5-0.1.txt-theta
collapsed-output-5-0.5-0.1.txt-trainll
collapsed-output-5-0.8-0.1.txt-phi
collapsed-output-5-0.8-0.1.txt-phi0
collapsed-output-5-0.8-0.1.txt-phi1
collapsed-output-5-0.8-0.1.txt-testll
collapsed-output-5-0.8-0.1.txt-theta
collapsed-output-5-0.8-0.1.txt-trainll


\section*{5 Blocked Gibbs Sampler}
\subsection*{5.1 Derivation}

\section*{6 Text Analysis}
\subsection*{6.1}

Refer to plots q6_p1_1.pdf, q6_p1_2.pdf, and q6_p1_3.pdf. We've plotted the test and training log likelihood as a function of
the number of iterations of iterations. Each plot exhibits the same trends; the training log likelihood increases over a period
of about 100 iterations, before leveling off, while the test log likelihood behaves asymptotically after about 50 iterations. The
log likelihood of the test data is much higher than that of the training log likelihood, because the test data is much smaller.

\subsection*{6.2}

Extra Credit

\subsection*{6.3}

Extra Credit

\subsection*{6.4}
The test likelihood increases with number of topics. The likelihood was highest for a topic size of 50.

TODO: show plot of num topics Vs final iteration test log-likelihood

\subsection*{6.4}

Refer to q6_p5.pdf for a plot of test log likelihood as a function of lambda. We see that as lambda decreases the log likelihood
increases. For instance, at lambda = 1.0 the log likelihood is about -2.28e+05, while at lambda = 0 the log likelihood is about
-2.255e+05.

\subsection*{6.6.a}
Below is an example of topics from ACL, NIPS and global that share a common theme:
ACL
Topic 1
model 0.016177074286
models 0.0115966992618
algorithm 0.0111221444474
probability 0.0100090151025
statistical 0.00885346456302
parsing 0.00821317593643
problem 0.00780716347197
based 0.00749694584482
given 0.00747310208966
data 0.00742426483454
using 0.00720128594085
maximum 0.00711986474806
grammar 0.00653818196435
probabilities 0.00591095459551
approach 0.00582103926486
algorithms 0.00580315705065
tree 0.00551910860789
time 0.00540066553474
grammars 0.00533389632103
framework 0.00530399171093
NIPS
Topic 0
algorithm 0.0218752931149
model 0.0147504470285
data 0.0144234378009
models 0.0110908058826
linear 0.010814378643
method 0.00987114130807
function 0.00961977446919
learning 0.00949037931016
problem 0.0088664442374
using 0.00817617120279
analysis 0.00789862326851
results 0.00742700154592
approach 0.00740190068817
non 0.00720331878952
probability 0.00708074495813
new 0.0069104242882
present 0.00653868103357
algorithms 0.00651221863588
gaussian 0.00629224913294
based 0.00609463801093
Global
Topic 0
algorithm 0.0180317637371
model 0.0153512978987
data 0.0119218228652
models 0.0113371360174
problem 0.00852501154491
linear 0.00821802077018
probability 0.00819562127406
method 0.00814143246541
using 0.00786159924427
learning 0.00731179331377
function 0.00725015695961
approach 0.00685902616221
based 0.00664327777912
statistical 0.00647947181681
results 0.00641035322625
algorithms 0.00628478063586
new 0.00618878561834
analysis 0.00603455553512
non 0.00576769211681
present 0.00565615967426

These topic themes seems most similar even for different number of topics. (K=10,20,30)
Another topic that was similar across all three corpora was:
ACL:
Topic 13
learning 0.0236569277439
data 0.0193323907822
training 0.0177041124081
performance 0.0159465697687
number 0.0124777968224
classification 0.0124619741943
task 0.0116708930056
features 0.011406526229
method 0.00993581517459
machine 0.00986731144034
problem 0.00986126749427
based 0.00981922475935
results 0.00958337539838
set 0.00951327638253
methods 0.00951126987734
tasks 0.00915523901263
classifiers 0.00896529773783
supervised 0.00872972939337
small 0.0081479403109
paper 0.00804743671283

NIPS:
Topic 13
data 0.0300396007649
learning 0.0297261962164
training 0.019597857133
method 0.0184479683093
classification 0.0171312523806
classifier 0.0145576037986
algorithms 0.0139714522147
performance 0.0133121610783
problem 0.0126331017797
set 0.0117619073234
class 0.0116415049338
methods 0.0116134797214
results 0.0116030737418
vector 0.011232450178
new 0.0111925377437
large 0.011071185214
based 0.0104201924844
number 0.0099513554648
paper 0.00978429669297
solution 0.00951159481878

Global:
Topic 13
learning 0.026646969666
data 0.0243572607502
training 0.0187586479328
performance 0.0149455191817
classification 0.0147060786892
method 0.0138767816174
number 0.0114753712323
problem 0.0112215436682
classifier 0.0108612392988
algorithms 0.0107930679265
set 0.0106309255965
results 0.010596472879
methods 0.0105545761066
based 0.0102011322731
task 0.00961196420842
large 0.00940082405471
features 0.00933764214712
vector 0.0092084187336
tasks 0.0091929993896
paper 0.00892055841457

We found some instances of where the global would match one of the corpus, but none where all three matched strongly.

When looking for different topics we found themes that were were different for each of the corpus, for example in the ACL corpus we found a topic on 'Dialogue Systems':
ACL:
Topic 6
dialogue 0.0219991347715
spoken 0.0177885134507
speech 0.0126581563494
task 0.0116648168309
human 0.0108525267634
domain 0.0104291922417
user 0.0104060649467
utterances 0.0101806663248
utterance 0.00997455934244
understanding 0.00996049578751
paper 0.00877467285834
recognition 0.00867095956506
using 0.00812891238502
based 0.00809590665096
model 0.00734018345985
dialogues 0.00718810758578
dialog 0.00689035087032
conversational 0.0068748601014
speaker 0.0066985660931
interaction 0.00637844847388

This theme was not found at all in the NIPS data set. This theme was not found in the global topics as well.
(It was mixed in with other themes as well). The closest theme to 'Dialogue Systems' in the global topics was:
Topic 6
control 0.0184446149204
dialogue 0.0143455278516
model 0.0136240531211
task 0.0132277819413
spoken 0.01159917392
feedback 0.0110393686706
using 0.010376825941
speech 0.00935091969147
based 0.00928811325642
human 0.0092401259526
domain 0.00913611264986
motor 0.00877570876671
paper 0.00842529991289
learning 0.00810564028211
goal 0.00771651501929
sensory 0.00754210125492
user 0.00741528821163
understanding 0.00707930464582
robot 0.00706626036798
utterances 0.00663944120127

But we can see they do not quite have the same theme. For example the global topic has the key words 'control','motor'
and 'robot' that do not appear in the ACL topic.
Similarly we seen topics in the NIPS data set that do not appear in the ACL or the global dat set.
This is one such example:
Topic 19
functions 0.0391002305843
number 0.0374894549313
linear 0.0369360971903
bound 0.028667229673
threshold 0.0282325753112
bounds 0.0257927722109
size 0.0252906042785
function 0.0250119924033
dimension 0.0235328870425
case 0.0189646286225
lower 0.0180149584335
vc 0.0164987927052
polynomial 0.0160152800684
loss 0.0121242195613
results 0.0119998061311
sigmoidal 0.0117484878921
upper 0.0116594383107
log 0.0113391769272
boolean 0.0108555791827
average 0.010223235153

There are several examples of NIPS and ACL being very different. For example we found topics in NIPS about image classification,
face recognition etc that were not in ACL. ACL had topics on Parsing, Dialogue systems that were not present in NIPS.

\subsection*{6.6.b}
We ran the topic model experiment with different values of lambda ranging from 0 to 1, in 0.25 increments.
When lambda was 0 we observed the following tendency:
A topic from the NIPS corpus would be very closely matching a topic from ACL which in turn would match a topic from the global topic set.
Even the topwords of the topics would show considerable overlap. Essential there was very little difference between global topics, NIPS
topics and ACL topics. Below is an example of this:
NIPS:
Topic 0
theory 0.0152752867533
considered 0.0146942165574
conditions 0.0141730129378
structure 0.0131071805393
type 0.0126643354912
basic 0.0123172919606
approach 0.0122303116898
potential 0.0109731030856
proposed 0.0105979079432
view 0.00938236943994
structures 0.00924778137613
general 0.00924354685752
framework 0.00911836278605
defined 0.00881891099982
representation 0.00875773682695
fact 0.00874619668056
assumptions 0.00849641891459
computational 0.00824488679022
make 0.0081200431694
expressions 0.00806720925643

ACL:
Topic 0
approach 0.00955759250749
structure 0.00932469521992
computational 0.00901337524993
theory 0.0088656224769
type 0.00868458508203
framework 0.00858519088188
view 0.00809793965069
fact 0.00804118005236
make 0.00802123819825
structures 0.00779098219251
idea 0.00766436554963
means 0.00642444603114
expressed 0.00636269110204
notion 0.0062751911379
way 0.0062170288999
form 0.0062164898045
aspects 0.00617640873327
proposed 0.00613470794857
problems 0.00611452805111
various 0.00609385621181

Global:
Topic 0
theory 0.0103994355748
approach 0.0103100396209
structure 0.0103069206994
type 0.00969403015477
computational 0.0090242737962
framework 0.00886248296989
view 0.00852344413744
fact 0.00834246413954
structures 0.00824754915285
make 0.00819565075453
idea 0.0077638351276
considered 0.00754218088409
basic 0.00748498173438
proposed 0.00720265283692
general 0.00682065830774
defined 0.00666832482311
way 0.00665240114992
various 0.00663051136196
means 0.00638142611106
representation 0.00634542296248

This makes sense because we always force our topic model select xdi = 0. Thus global counts were always used to select topic of a word.

The exact opposite was noticed when lambda = 1. In this case xdi = 1 and this forces the model to choose zdi from topic specific counts only. This makes the corpus-dependent topics as distinct as possible. However, the global topics were combination of the 2 corpus based topics. We show this observation with an example below:

NIPS:
Topic 1
learning 0.053560952863
algorithm 0.0476659197163
method 0.0291401253165
function 0.0251698271345
gradient 0.0218832202456
new 0.019393991553
algorithms 0.0190215622899
based 0.0176015626806
results 0.0161289210082
present 0.013347054748
convergence 0.0130211751088
stochastic 0.012943928966
descent 0.0128555929271
optimal 0.0125686866905
problems 0.0123940403841
line 0.0112869986241
framework 0.00998514068744
simple 0.00986540714555
entropy 0.00983279638515
class 0.00977730839018
ACL:
Topic 1
disambiguation 0.0233493060825
sense 0.0227870932636
word 0.0218888237831
noun 0.0159414855324
words 0.0148539009865
corpus 0.0137351704195
verb 0.0133779037281
syntactic 0.0132941125277
types 0.0123804840712
different 0.0120352561255
possible 0.0113540661016
classes 0.0112404308713
ambiguous 0.0106116272386
nouns 0.0103225790277
ambiguity 0.00968102738332
contexts 0.00946866498308
relations 0.00944057161089
wordnet 0.00915839507787
senses 0.0090816674575
construction 0.00893322583557
Global:
Topic 1
learning 0.0282249707329
algorithm 0.0255696975806
method 0.0153599667806
function 0.0133475377151
gradient 0.0115144534703
disambiguation 0.0114830826619
sense 0.0112948544456
word 0.0107859221618
new 0.0102070479895
algorithms 0.0100354918429
based 0.0100022010496
results 0.00858126571942
noun 0.00784657195485
words 0.00731867510807
present 0.00701697279648
convergence 0.00684864181333
stochastic 0.00680968856066
descent 0.0067635072899
corpus 0.0067547123613
optimal 0.00661169499662

We can see that the ACL topic and NIPS topichave nothing similar at all. There is not even one keyword shared across them. The Global topic that was closed to either of them was a combination of the 2 of these topics.Thus setting setting lambda = 1 forces the corpus-dependent topics to be as disticnt as possible, and reducing lambda allows the corpus-dependent topics to share some attributes from the global topics.

\subsection*{6.6.c}
When alpha was small (0.001) we noticed that the global topics closely matched one topic in either NIPS or ACL but not both.
The effect was very similar to having a large lamda value, in that the 2 corpus specific topics did not share any commanlities. But one difference was in the fact that even the global topics closely corresponded to one of the corpus specic topics. Here is an example:
NIPS:
Topic 5
images 0.0410480607618
image 0.0258754385985
face 0.0225305751957
facial 0.0202748259436
recognition 0.0153357250119
wavelet 0.014342053533
sdm 0.0116074738546
faces 0.0111449598949
processing 0.0109342063738
compression 0.0100716081724
natural 0.00988376080069
gestures 0.00870922819804
resource 0.0086971774364
ability 0.00869219925919
video 0.0085884670679
doing 0.00835519546142
multiple 0.00821448159175
accurately 0.00788600302408
performed 0.00779761574339
text 0.00717814610869
ACL:
Topic 5
information 0.00946530556189
text 0.008634625417
language 0.0075118067353
paper 0.007269718655
introduction 0.00723863680093
retrieval 0.0069269789511
documents 0.00678145834625
using 0.00561962725638
results 0.00536930274001
question 0.00532608668089
based 0.0052660817528
large 0.00498759939707
new 0.00481660002936
processing 0.00476814240649
document 0.00476758783097
natural 0.00472729426331
research 0.00468625554802
extraction 0.00466325593758
systems 0.00438320994137
analysis 0.00426115545668

Global:
Topic 5
information 0.0090226221977
text 0.00860023835746
language 0.00737593761979
paper 0.00711012836669
retrieval 0.00693653602072
introduction 0.0067104414601
documents 0.00628681970329
using 0.00549210942081
results 0.00542707527462
based 0.00535642452161
processing 0.00533014376626
natural 0.0052034120436
question 0.00493768779341
new 0.00488974578961
large 0.00466275996326
research 0.00448949200962
systems 0.00444909407008
analysis 0.00444774794064
extraction 0.00443890834317
document 0.00442010954034

The above 3 topics show that the global closely matched the ACL topic. We did not find any topic in the global topic set than matched the NIPS Topic 5.
Similarly, here is an example where the global topic matched the NIPS topic but did not match any ACL topic.
NIPS:
Topic 6
networks 0.0400012223316
layer 0.0355832546038
hidden 0.0340426850619
units 0.0297327302045
neural 0.0292441935316
time 0.0236760845029
network 0.0214810194423
threshold 0.0163439719183
depth 0.0120301950234
net 0.0119087822162
regions 0.0106863690813
internal 0.0105521766542
ann 0.0104753708873
weights 0.0103834807446
functions 0.0103384330005
unit 0.0102672858853
capacity 0.00954105663512
layered 0.00938272990112
work 0.0093664322053
multi 0.00902116785747
ACL:
Topic 6
overall 0.0104036609725
time 0.00934432229404
view 0.00913086949803
multi 0.00863859168453
earlier 0.00845128227215
unit 0.00780469079382
instances 0.00734662235475
quantity 0.00699236194259
tree 0.0068684134886
subsequent 0.00684258930905
internal 0.00676967127057
self 0.00652255996235
indicate 0.00643422992496
separate 0.00633665360426
units 0.00614692963206
chain 0.00599295090074
addition 0.00593501988839
size 0.00589847514123
chains 0.00573069610148
single 0.00572209536948
Global:
Topic 6
networks 0.0283205662323
layer 0.0270968321066
hidden 0.0243711391275
units 0.0231815938231
neural 0.0217767030868
time 0.0200543566135
network 0.0163088941686
threshold 0.0126129711102
unit 0.0100145444424
internal 0.0098737454813
depth 0.00977923378925
multi 0.00941200023128
net 0.00919550155843
weights 0.00906039029932
work 0.00854187801695
functions 0.0079867046736
regions 0.00756635924387
representation 0.0075598236236
capacity 0.00744685417727
ann 0.00739913637493

In the case of very high smoothing e.g. alpha = 10.0 we noticed that all 3 topic sets (NIPS, ACL and Global) looked very similar. It was like in the case where lambda = 0.
We have attached topwords for each of these combinations.

When the beta parameter was tweaked, we noticed the topics changing in a different way. with very high values of beta, we noticed that topwords for topics were not very descriptive, this in turn make topic themes not very clear. This was noticed for both global topics and corpus-dependent topics. Also the topword weights in the topics were lower than in typical cases. When beta was high, it was hard to tell which corpus a topic set came from i.e. whether it was from the NIPS corpus or ACL. This is because the beta parameter smooths words heavily that even low frequency words get high weights. The opposite was observed when beta was made small. The topwords for high and low beta values are attached.

% EXAMPLE PROOF
%\begin{proof}
%  Let $t,u \in \mathbb{R}$ where $t=xy$ and $u=zw$. So,
%  \begin{align*}
%    4xyzw &= 2\cdot2tu \\
%    &\le 2\cdot(t^2+u^2) \\
%    &= 2\cdot((xy)^2+(zw)^2) &&\text{(substituting variables)} \\
%    &= 2\cdot(x^2y^2+z^2w^2) \\
%    &= 2x^2y^2+2z^2w^2 \\
%    &\le ((x^2)^2+(y^2)^2)+((z^2)^2)+(w^2)^2) \\
%    &= x^4+y^4+z^4+w^4 &&\qedhere
%  \end{align*}
%\end{proof}

\end{document}
